\documentclass[a4paper, 13pt, twocolumn, landscape]{book}

\usepackage[left=1cm, right=1cm, bottom=1cm, top=1cm, headsep=0.2cm]{geometry}
\usepackage{array}
\usepackage{graphicx}
\usepackage{pdflscape}
\usepackage{multicol}
\usepackage{fancyhdr}
\fancyhead{}
\fancyfoot{}
\lhead{INSTITUTO MILITAR DE ENGENHARIA - \the\year}
\chead{LOREM IPSUM}
\rhead{\thepage}

\renewcommand{\headrulewidth}{2pt}

\usepackage{xcolor}
\usepackage{listings}
\usepackage{listingsutf8}
\definecolor{verdecpp}{rgb}{0,0.5,0}
\lstset{
  language=C++,
  inputencoding=utf8/latin1,
  alsolanguage={[Sharp]C},
  basicstyle=\ttfamily\small,
  keywordstyle=\color{blue},
  stringstyle=\color{verdecpp},
  commentstyle=\color{red},
  extendedchars=true,
  showspaces=false,
  showstringspaces=false,
  %numbers=left,
  numberstyle=\tiny,
  breaklines=true,
  %backgroundcolor=\color{green!10},
  breakautoindent=true,
  captionpos=b,
  xleftmargin=0pt,
}

\begin{document}

  \pagestyle{fancy}

  \setlength{\columnseprule}{1pt}
  \large

  % Graphs
  \textbf{DFS}
  \lstinputlisting[language=C++]{graphs/dfs.cpp}
  \textbf{BFS}
  \lstinputlisting[language=C++]{graphs/bfs.cpp}
  \textbf{Topological Sort}
  \lstinputlisting[language=C++]{graphs/kahn.cpp}
  \textbf{Articulation points and bridges}
  \lstinputlisting[language=C++]{graphs/articulation.cpp}
  \textbf{Strongly Connected Components}
  \lstinputlisting[language=C++]{graphs/kosaraju.cpp}
  \textbf{Minimum Spanning Tree}
  \lstinputlisting[language=C++]{graphs/kruskal.cpp}
  \lstinputlisting[language=C++]{graphs/prim.cpp}
  \textbf{Shortest path}
  \lstinputlisting[language=C++]{graphs/dijkstra.cpp}
  \lstinputlisting[language=C++]{graphs/bellman-ford.cpp}
  \lstinputlisting[language=C++]{graphs/spfa.cpp}
  \lstinputlisting[language=C++]{graphs/floyd-warshall.cpp}
  %\textbf{Network Flow}
  %\lstinputlisting[language=C++]{graphs/.cpp} % Flow
  \textbf{Maximum Bipartite Cardinality Matching}
  \lstinputlisting[language=C++]{graphs/kuhn.cpp}
  \textbf{Lowest Common Ancestor}
  \lstinputlisting[language=C++]{graphs/lca.cpp}

  % Math

  \textbf{Number Theory}
  \lstinputlisting[language=C++]{math/basics.cpp}
  \textbf{Primes}
  \lstinputlisting[language=C++]{math/sieve.cpp}
  \textbf{Euler phi}
  \lstinputlisting[language=C++]{math/euler-phi.cpp}
  \textbf{Extended Euclidean}
  \lstinputlisting[language=C++]{math/extended-euclid.cpp}
  \textbf{Multiplicative Inverse}
  \lstinputlisting[language=C++]{math/mult-inverse.cpp}
  \textbf{Prime factors}
  \lstinputlisting[language=C++]{math/prime-factors.cpp}
  \lstinputlisting[language=C++]{math/pollard-rho.cpp}
  \textbf{Primarily test}
  \lstinputlisting[language=C++]{math/miller-rabin.cpp}

  % Strings

  \textbf{String Matching}
  \lstinputlisting[language=C++]{strings/rabin-karp.cpp}
  \lstinputlisting[language=C++]{strings/kmp.cpp}
  \lstinputlisting[language=C++]{strings/suffix-array.cpp}

  % Data Structures

  \textbf{Disjoint Set Union}
  \lstinputlisting[language=C++]{data-structures/dsu.cpp}
  \textbf{Fenwick Tree}
  \lstinputlisting[language=C++]{data-structures/bit.cpp}
  \textbf{Segment Tree}
  \lstinputlisting[language=C++]{data-structures/segtree.cpp}
  \textbf{Heavy-Light Decomposition}
  \lstinputlisting[language=C++]{data-structures/hld.cpp}

  \newpage
  \onecolumn
  \begin{table}[ht]
    \centering
    \resizebox{\textwidth}{!}{
      \begin{tabular}{|c|c|c|c|m{80mm}|m{120mm}|c|}
        \hline
        & A & C & N & Assunto & Descricao & Diff\\
        \hline
        & & & & & &\\
        A & & & & & &\\
        & & & & & &\\
        \hline
        & & & & & &\\
        B & & & & & &\\
        & & & & & &\\
        \hline
        & & & & & &\\
        C & & & & & &\\
        & & & & & &\\
        \hline
        & & & & & &\\
        D & & & & & &\\
        & & & & & &\\
        \hline
        & & & & & &\\
        E & & & & & &\\
        & & & & & &\\
        \hline
        & & & & & &\\
        F & & & & & &\\
        & & & & & &\\
        \hline
        & & & & & &\\
        G & & & & & &\\
        & & & & & &\\
        \hline
        & & & & & &\\
        H & & & & & &\\
        & & & & & &\\
        \hline
        & & & & & &\\
        I & & & & & &\\
        & & & & & &\\
        \hline
        & & & & & &\\
        J & & & & & &\\
        & & & & & &\\
        \hline
        & & & & & &\\
        K & & & & & &\\
        & & & & & &\\
        \hline
        & & & & & &\\
        L & & & & & &\\
        & & & & & &\\
        \hline
      \end{tabular}
    }
  \end{table}

\end{document}
